\def\mytitle{BCD TO EXCESS 3 IN VAMAN ESP}
\def\myauthor{TOTLI VARSHA REDDY}
\def\contact{varshareddy724@gmail.com}
\def\mymodule{Future Wireless Communication (FWC)}
\documentclass[10pt, a4paper]{article}
\usepackage[a4paper,outer=1.5cm,inner=1.5cm,top=1.75cm,bottom=1.5cm]{geometry}
\twocolumn
\usepackage{graphicx}
\graphicspath{{./images/}}
\usepackage[colorlinks,linkcolor={black},citecolor={blue!80!black},urlcolor={blue!80!black}]{hyperref}
\usepackage[parfill]{parskip}
\usepackage{lmodern}
\usepackage{tikz}
%\documentclass[tikz, border=2mm]{standalone}
\usepackage{karnaugh-map}
%\documentclass{article}
\usepackage{tabularx}
\usepackage{circuitikz}
\usetikzlibrary{calc}
\usepackage{enumitem}
\usepackage{kvmap}
\renewcommand*\familydefault{\sfdefault}
\usepackage{watermark}
\usepackage{lipsum}
\usepackage{xcolor}
\usepackage{listings}
\usepackage{float}
\usepackage{titlesec}
       \usepackage[latin1]{inputenc}
       \usepackage{color}
       \usepackage{array}
       \usepackage{longtable}
       \usepackage{calc}
       \usepackage{multirow}
       \usepackage{hhline}
       \usepackage{ifthen}

\titlespacing{\subsection}{1pt}{\parskip}{3pt}
\titlespacing{\subsubsection}{0pt}{\parskip}{-\parskip}
\titlespacing{\paragraph}{0pt}{\parskip}{\parskip}
\newcommand{\figuremacro}[5]{
    
}

\lstset{
frame=single, 
breaklines=true,
columns=fullflexible
}

\def\ifundefined#1{\expandafter\ifx\csname#1\endcsname\relax}
\ifundefined{inputGnumericTable}
\def\gnumericTableEnd{\end{document}}
\else
   \def\gnumericTableEnd{}
\fi
\providecommand{\gnumericmathit}[1]{#1} 
\providecommand{\gnumericPB}[1]%
{\let\gnumericTemp=\\#1\let\\=\gnumericTemp\hspace{0pt}}
 \ifundefined{gnumericTableWidthDefined}
        \newlength{\gnumericTableWidth}
        \newlength{\gnumericTableWidthComplete}
        \newlength{\gnumericMultiRowLength}
        \global\def\gnumericTableWidthDefined{}
 \fi
 \ifthenelse{\isundefined{\languageshorthands}}{}{\languageshorthands{english}}
\providecommand\gnumbox{\makebox[0pt]}
\setlength{\bigstrutjot}{\jot}
\setlength{\extrarowheight}{\doublerulesep}
\setlongtables

\setlength\gnumericTableWidth{%
	98pt+%
	118pt+%
0pt}
\def\gumericNumCols{2}
\setlength\gnumericTableWidthComplete{\gnumericTableWidth+%
         \tabcolsep*\gumericNumCols*2+\arrayrulewidth*\gumericNumCols}
\ifthenelse{\lengthtest{\gnumericTableWidthComplete > \linewidth}}%
         {\def\gnumericScale{1*\ratio{\linewidth-%
                        \tabcolsep*\gumericNumCols*2-%
                        \arrayrulewidth*\gumericNumCols}%
{\gnumericTableWidth}}}%
{\def\gnumericScale{1}}

\ifthenelse{\isundefined{\gnumericColA}}{\newlength{\gnumericColA}}{}\settowidth{\gnumericColA}{\begin{tabular}{@{}p{98pt*\gnumericScale}@{}}x\end{tabular}}
\ifthenelse{\isundefined{\gnumericColB}}{\newlength{\gnumericColB}}{}\settowidth{\gnumericColB}{\begin{tabular}{@{}p{118pt*\gnumericScale}@{}}x\end{tabular}}

%\thiswatermark{\centering \put(181,-119.0){\includegraphics[scale=0.13]{iith_logo3}} }
\title{\mytitle}
\author{\myauthor\hspace{1em}\\\contact\\FWC22038\hspace{6.5em}IITH\hspace{0.5em}\mymodule\hspace{6em}ASSIGNMENT-10}
\begin{document}
	\maketitle
	\tableofcontents
	\begin{abstract}
	      This manual shows how to represent the K-MAP for  expressions for the function W,X,Y,Z shown in below truth table. \\
 
 \centering
  \begin{center}
 \begin{tabular}{ |c |c |c |c |}

 \hline
\textbf{DECIMAL} & {BCD CODE} & {EXCESS3} & {}\\
\hline
\textbf{Digit} & {A B C D}  & {W X Y Z} & {a b c d e f g}\\
\hline
\textbf{0} & {0  0  0 0} & {0  0  1 1} & {0 0 0 0 1 1 0}\\
\textbf{1} & {0  0  0 1} & {0  1  0 0} & {1 0 0 1 1 0 0}\\
\textbf{2} & {0  0  1 0} & {0  1  0 1} & {0 1 0 0 1 0 0}\\
\textbf{3} & {0  0  1 1} & {0  1  1 0} & {0 1 0 0 0 0 0}\\
\textbf{4} & {0  1  0 0} & {0  1  1 1} & {0 0 0 1 1 1 1}\\
\textbf{5} & {0  1  0 1} & {1  0  0 0} & {0 0 0 0 0 0 0}\\
\textbf{6} & {0  1  1 0} & {1  0  0 1} & {0 0 0 1 1 0 0}\\
\textbf{7} & {0  1  1 1} & {1  0  1 0} & {0 0 0 1 0 0 0}\\
\textbf{8} & {1  0  0 0} & {1  0  1 1} & {0 0 0 0 0 0 0}\\
\textbf{9} & {1  0  0 1} & {1  1  0 0} & {0 1 1 0 0 0 1}\\
\hline

 
 \end{tabular}
 \end{center}
	  	\end{abstract}
	  	
	

	\section{Components}
  \begin{tabularx}{0.48\textwidth} { 
  | >{\centering\arraybackslash}X 
  | >{\centering\arraybackslash}X 
  | >{\centering\arraybackslash}X | }
\hline
 \textbf{Components}& \textbf{Values} & \textbf{Quantity}\\
\hline
Vaman Board&  & 1 \\  
\hline
JumperWires& M-F& 5 \\ 
\hline
Breadboard &  & 1 \\
\hline
USB-C Cable &  & 1 \\
\hline
USB-UART &  & 1 \\
\hline
\end{tabularx}
   \section{Implementation}
%  \section{Implementation}
   	\paragraph{}
\textbf{KMAP FOR EQUATIONS}
\newline
\newline
\centering
\begin{kvmap}
\begin{kvmatrix}{C,D,A,B}
0&0&0&1 \\
0&1&0&1 \\
0&1&0&0 \\
0&1&0&0 \\
\end{kvmatrix}
\bundle[color=red]{3}{0}{3}{1}
\bundle[color=red]{1}{1}{1}{2}
\bundle[color=cyan]{1}{2}{1}{3}
\end{kvmap}
%\begin{equation}
\centering
\newline
W= AB'C'+A'BD+A'BC
%\end{equation}
\newline
\centering
\begin{kvmap}
\begin{kvmatrix}{C,D,A,B}
0&1&0&0 \\
1&0&0&1 \\
1&0&0&0 \\
1&0&0&0 \\
\end{kvmatrix}
\bundle[color=red]{0}{1}{0}{2}
\bundle[color=red]{0}{2}{0}{3}
\bundle[color=cyan]{1}{0}{1}{0}
\bundle[color=blue]{3}{1}{3}{1}
\end{kvmap}
\begin{equation}
X= A'B'D+A'B'C+A'BC'D'+AB'C'D
\end{equation}
\centering
\begin{kvmap}
\begin{kvmatrix}{C,D,A,B}
1&1&0&1 \\
0&0&0&0 \\
1&1&0&0 \\
0&0&0&0 \\
\end{kvmatrix}
\bundle[color=red]{0}{0}{1}{0}
\bundle[color=blue]{0}{2}{1}{2}
\bundle[color=cyan]{3}{0}{3}{0}
\end{kvmap}
\begin{equation}
Y=A'C'D'+A'C'D+AB'C'D'
\end{equation}
\centering
\begin{kvmap}
\begin{kvmatrix}{C,D,A,B}
1&1&0&1 \\
0&0&0&0 \\
0&0&0&0 \\
1&1&0&0 \\
\end{kvmatrix}
\bundle[color=red, invert=true,overlapmargins=8pt]{0}{0}{1}{3}
\bundle[color=cyan]{3}{0}{3}{0}
\end{kvmap}
\begin{equation}
Z= A'D'+AB'C'D'
\end{equation}

    \paragraph{Karnugh Map :}
  
The code below realizes the Boolean logic for W,X,Y,Z  using 5V,GND of Vaman Board.\\
2,3,4,5,6,7,8 GPIO Pins of Vaman Board are configured as input pins and the required Logic for U,V,W are drawn from 5V (Digital '1'),GND (Digital '0'). Built in led will glow based on G satisfying the Table
\subsection{The steps for implementation:}
\begin{enumerate}
\item Connect the USB-UART pins to the Vaman ESP32 pins according to Table 

\begin{tabular}[c]{%
	b{\gnumericColA}%
	b{\gnumericColB}%
	}
\hhline{|-|-}
	 \multicolumn{1}{|p{\gnumericColA}|}%
	{\gnumericPB{\centering}\gnumbox{{\color[rgb]{0.79,0.13,0.12} VAMAN LC PINS}}}
	&\multicolumn{1}{p{\gnumericColB}|}%
	{\gnumericPB{\centering}\gnumbox{{\color[rgb]{0.79,0.13,0.12} UART PINS}}}
\\
\hhline{|--|}
	 \multicolumn{1}{|p{\gnumericColA}|}%
	{\gnumericPB{\centering}\gnumbox{GND}}
	&\multicolumn{1}{p{\gnumericColB}|}%
	{\gnumericPB{\centering}\gnumbox{GND}}
\\
\hhline{|--|}
	 \multicolumn{1}{|p{\gnumericColA}|}%
	{\gnumericPB{\centering}\gnumbox{ENB}}
	&\multicolumn{1}{p{\gnumericColB}|}%
	{\gnumericPB{\centering}\gnumbox{ENB}}
\\
\hhline{|--|}
	 \multicolumn{1}{|p{\gnumericColA}|}%
	{\gnumericPB{\centering}\gnumbox{TXD0}}
	&\multicolumn{1}{p{\gnumericColB}|}%
	{\gnumericPB{\centering}\gnumbox{RXD}}
\\
\hhline{|--|}
	 \multicolumn{1}{|p{\gnumericColA}|}%
	{\gnumericPB{\centering}\gnumbox{RXD0}}
	&\multicolumn{1}{p{\gnumericColB}|}%
	{\gnumericPB{\centering}\gnumbox{TXD}}
\\
\hhline{|--|}
	 \multicolumn{1}{|p{\gnumericColA}|}%
	{\gnumericPB{\centering}\gnumbox{0}}
	&\multicolumn{1}{p{\gnumericColB}|}%
	{\gnumericPB{\centering}\gnumbox{IO0}}
\\
\hhline{|--|}
	 \multicolumn{1}{|p{\gnumericColA}|}%
	{\gnumericPB{\centering}\gnumbox{5V}}
	&\multicolumn{1}{p{\gnumericColB}|}%
	{\gnumericPB{\centering}\gnumbox{5V}}
\\
\hhline{|-|-|}
\end{tabular}
 \item Flash the following setup code through USB-UART using laptop
\begin{center}
\fbox{\parbox{8cm}{\url{https://github.com/9705701645/FWC/blob/main/iot/codes/setup/src/main.cpp}}}
\end{center}
\begin{center}
\end{center}
\begin{lstlisting}
svn co https://github.com/9705701645/FWC/trunk/iot/codes/setup
cd  setup
pio run
pio run -t upload
\end{lstlisting}

after entering your wifi username and password (in quotes below)
\begin{lstlisting}
#define STASSID "..." // Add your network credentials
#define STAPSK  "..."
\end{lstlisting}
in src/main.cpp file
\item You can notice that vaman will be connnected to the network credentials provided above.Connect your laptop to the same network ,You should be able to find the ip address of your vaman-esp on laptop using 
\begin{lstlisting}
ifconfig
nmap -sn 192.168.6.1/24
\end{lstlisting}
where your computer's ip address is the output of ifconfig and given by 192.168.6.x
\item Login to termux-ubuntu on the android device and execute the following commands:
\begin{lstlisting}
proot-distro login debian
cd  /data/data/com.termux/files/home/
mkdir iot
svn co https://github.com/9705701645/FWC/trunk/iot/codes/ota
cd codes
\end{lstlisting}
\item Assuming that the username is krishna and password is 123, flash the following code wirelessly
\begin{center}
\fbox{\parbox{8cm}{\url{https://github.com/9705701645/FWC/blob/main/iot/codes/ota/src/main.cpp}}}
\end{center}
through 
\begin{lstlisting}
pio run 
pio run -t nobuild -t upload --upload-port ip_addres_of_esp
\end{lstlisting}
where you may replace the above ip address with the ip address of your vaman-esp.
\end{enumerate}
\end{document}
